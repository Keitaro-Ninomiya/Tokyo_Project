\item I find workers with a co-worker drafted in 1940 tend to stay longer in the institution in 1943 (Table 2) (*All the results here are statistically insignificant, but I predict this is due to the test being underpowered. I'm currently adding new data to the dataset by digitizing directory records for 1938-1941.). Column 1 in the table shows the effect on workers in the same office as the draftee. Workers in drafted offices were 16\% more likely to remain hired in the subsequent period. Column 2 redefines the treatment of workers with the same occupation as the draftee. The coefficient reduces in magnitude to 12\%. I infer workers in other occupations of the draftee learned more about the job content of draftees' occupation, raising their likelihood of being retained in future periods.

Column 3 in Table 2 shows how the chances of job retention differs across offices with different skill sets. $EngRatio$ measures the proportion of engineers to non-engineers in a given office. I find offices with more engineers tends to have a worker retained. Offices with a small number of non-engineers created a strong need for non-engineers to learn about job tasks related to engineering, raising the chances of a non-engineer to retain his job.
\input{Results/Table2}